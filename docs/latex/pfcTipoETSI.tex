%:Clase del documento
\documentclass[fontsize=10pt, Myfinal=true, twoside, numbers=noenddot]{scrbook}
%Minion=true, English=true, Myfinal=true

%:Paquete de estilos propuesto
\usepackage{libroETSI}
 
\usepackage{showexpl}

%:Paquete específico para cargar tikz (y sus librerías) y pgfplots
\usepackage{dtsc-creafig}

%:Paquete para notaciones específicas
\usepackage{notacion}

%:Paquete para incorporar aspectos concretos de la edición
\usepackage{edicionPFC}

%:Para modificar fácilmente la fuente del texto.
\makeatletter
\ifdtsc@Minion % Queremos utilizar la fuente Minion y lo hemos declarado al principio
	\ifluatex
		\setmainfont[Renderer=Basic, Ligatures=TeX,	% Fuente del texto 
		Scale=1.01,
		]{Minion Pro}
   		% En este caso conviene modificar ligeramente el tamaño de las fuentes matemáticas
		\DeclareMathSizes{10}{10.5}{7.35}{5.25}
		\DeclareMathSizes{10.95}{11.55}{8.08}{5.77}
		\DeclareMathSizes{12}{12.6}{8.82}{6.3}
%		\setmainfont[Renderer=Basic, Ligatures=TeX,	% Fuente del texto 
%		]{Adobe Garamond Pro}
%		\setmainfont[Renderer=Basic, Ligatures=TeX,	% Fuente del texto 
%		]{Palatino LT Std}
	\fi
\else
	\ifluatex
		% Para utilizar la fuente Times New Roman, o alguna otra que se tenga instalada
		\setmainfont[Renderer=Basic, Ligatures=TeX,	% Fuente del texto 
		Scale=1.0,
		]{Times New Roman}
	\else
		\usepackage{tgtermes} 	%clone of Times
		%\usepackage[default]{droidserif}
		%\usepackage{anttor} 	
	\fi
\fi
\makeatother

%Por si quieren usar bibliografía con BIBER
%BIBER%%:Para la bibliografía en BIBER, descomentar las líneas siguientes
%\defbibheading{etsi}[]{%
%	\chapter*{Bibliografía}%
%	\chaptermark{Bibliografía} 
%	\markboth{#1}{#1}}
%\addbibresource{bibliografiaLibroETSI.bib}

% Ejemplo de Glosario
\newacronym[type=main]{ETSI}{ETSI}{Escuela Técnica Superior de Ingeniería}
\newacronym[type=main]{US}{US}{Universidad de Sevilla}
\newacronym[type=main]{DMC}{DMC}{Canal Discreto sin Memoria}


\makeindex
\makeglossaries %Si no se quiere el glosario, comentar esta línea.

% Formato A4
\geometry
{paperheight=297mm,%
paperwidth=210mm,%
top=25mm,%
headsep=8.5mm,%
includefoot, 
textheight=240mm, 
textwidth=150mm, 
bindingoffset=0mm, 
twoside}

\usepackage[a4,center]{crop}%para poner las cruces de esquina de página, poner la opción cross

%:Esquema de numeración por defecto
\setenumerate[1]{label=\normalfont\bfseries{\arabic*.}, leftmargin=*, labelindent=\parindent}
\setenumerate[2]{label=\normalfont\bfseries{\alph*}), leftmargin=*}
\setenumerate[3]{label=\normalfont\bfseries{\roman*.}, leftmargin=*}
\setlist{itemsep=.1em}
\setlength{\parindent}{1.0 em}

\setcounter{tocdepth}{4}						% El nivel hasta el que se muestra el índice 


%:Empieza el documento

\begin{document}
%:Para incluir toda la referencia bibliográfica aunque no se cite, descomente la siguiente línea
%\nocite{*}


%PORTADA
%ver edicionPFC.sty para modificaciones

%:Para crear la portada y la portada interior (pagina titular)
\titulo{Control de un robot de 3GDL mediante visual servoing } %\mbox evita que se divida una palabra al cambiar de línea
\autor{Richard M. Haes-Ellis}
\director{Ignacio Alvarado Aldea}
\titulodirector{Profesor Titular}

\departamento{Dpto. Ingeniería de Sistemas y Automática}
\centro{Escuela Técnica Superior de Ingeniería}
\universidad{Universidad de Sevilla}
\titulacion{Ingeniería Electrónica, Robótica y Mecatrónica}
\fecha{2019}
\nombretrabajo{Proyecto Fin de Carrera} %Trabajo Fin de Grado, Proyecto fin de Máster,....

\hypersetup
	{
 	linkcolor=black, %Tocar para poner color en enlaces
	pdfauthor={\elautor},
	pdftitle={\nombretrabajo,\eltitulo}, 
	pdfkeywords={Latex, edición, formato de texto}	
	}

\portadaPFC{figuras/LogoUS.pdf}{figuras/LogoTSC.pdf} %logo de la Universidad y logo del departamento, si lo hubiera. Para cambiar el pie de página con los logos, debe editarse el fichero ediciónPFC.sty

%Fin Portada

%:Todo lo que constituye la primera parte del libro que no es el cuerpo del libro en realidad
\frontmatter
\pagenumbering{Roman} %Pone la numeración en mayúscula (En español parece que es obligatorio)

%\include{dedicatoria/dedicatoria}%¿Comentar para proyectos/tesis?
% !TEX root =../LibroTipoETSI.tex
\chapter*{Agradecimientos}
%\pagestyle{especial}
\pagestyle{empty}
%\chaptermark{Agradecimientos}
\phantomsection
%\addcontentsline{toc}{listasf}{Agradecimientos}
%\vspace{1cm}
%{\huge{Agradecimientos}}
%\vspace{1cm}

\lettrine[lraise=-0.1, lines=2, loversize=0.25] Este trabajo refleja la ilusión y motivación que he tenido a lo largo de mi carrera
académica, sin embargo esta etapa ha sido posible gracias a una persona muy importante, mi madre.

A mi madre por salirse de su camino para darme la mejor calidad de vida que podría tener, por recorrer medio mundo en busca de un médico que pudiese devolverme
mi capacidad de audición, por estar al lado mío en los momentos más difíciles, por sacrificarse incontables horas para que pudiese estudiar donde quisiera, de
dar un paso más donde la mayoría habrían parado, por asegurarse de que tuviese las mismas oportunidades que el resto, por su constante e inmenso apoyo. Mamá, gracias
por animarme a dar lo mejor de mí, por ayudarme a levantarme cuando tropezaba, gracias por tu cariño y apoyo sin pedir nunca nada a cambio. 

También quiero dar las gracias a mi hermano Thomas, por apoyarme durante mis estudios e inspirar este proyecto, por hacernos reír a mi y a mi madre cuando
los tiempos se vuelven difíciles, por ser mi mejor amigo y poder contar contigo para todo.

Quisiera agradecer también a mi tutor, Ignacio Alvarado, por la oportunidad que se me brindó para ayudar con los cursos
de Arduino de la escuela y acercar a la comunidad universitaria el mundo maker. Gracias por aportar tu tiempo y esfuerzo en este proyecto.

A mis profesores de la escuela que me han enseñado a ver la ingeniería con una ilusión y motivación en el que todos colaboramos para hacer un mundo mejor.

También quiero darles las gracias a esos compañeros de clase que se han convertido en amigos, y que
sin saberlo, consiguieron que la vida tuviera más sabor durante los años de duro estudio. Darle las gracias a esas personas
que con los que he compartido incontables horas en las salas de estudio de medicina y por último
gracias a todos aquellos que tuvieron la amabilidad y la paciencia para guiarme cuando les pedí ayuda.
%gradecemos}, a todos nuestros maestros, cuanto nos enseñaron.

{\flushleft{\hfill \emph{Richard Mark Haes-Ellis}}}%
{\flushleft{\hfill \emph{Sevilla, 2020}}}%

%PFC/PFM/TESIS
% !TEX root =../LibroTipoETSI.tex
\chapter*{Resumen}
\pagestyle{especial}
\chaptermark{Resumen}
\phantomsection
\addcontentsline{toc}{listasf}{Resumen}

\lettrine[lraise=-0.1, lines=2, loversize=0.2]{E}{n} este proyecto se presenta un robot de 3 grados de libertad compuesto por 
un mecanismo tipo \textit{"pan-tilt"} montado encima de un rail permitiendo su desplazamiento lateral. Este robot es capaz 
de visualizar mediante una cámara 2D cualquier objeto en su campo de visión y realizar un seguimiento del mismo, centrando  
el objetivo en el centro de la imagen. Esto se consigue controlando los motores mediante un control PID, donde la señal 
a controlar es la posición del objeto en la imagen y el error es la diferencia entre esta posición el centro de la imagen.

Para la realización de este trabajo se echaba en falta un robot como el descrito anteriormente, por tanto, se tomo la iniciativa 
de diseñar, simular en CAD, fabricar y ensamblar el robot a medida para esta aplicación. Se entra en profundidad en el diseño mecánico 
explicando el funcionamento de las diferentes articulaciones y se explica también el sistema electrónico que maneja los motores y 
sensores del sistema. Además se ha llevado a cabo la programación del control de los motores mediante interrupciones y timers, permitiendo 
un funcionamento eficiente del sistema embebido. 

Como es de saber, la tarea de seguimiento de objetos mediante visión artificial es costosa a nivel de computación, por tanto se ha 
realizado los programas de tracking en un PC para obtener un rendimiento aceptable. Aqui surge la necesidad de comunicar de forma rápida 
y segura, el microcontrolador y el PC. Esta comunicación se implemento usando una máquina de estados que permite manejar paquetes de 
información que provienen del PC. 

Con esto solo queda el control. El control se implmenetó en el PC en el mismo bucle que realiza el seguimiento del objeto, aqui hemos ajustado 
las ganacias $Kp$,$Ki$,$Kd$ experimentalmente para obtener una buena respuesta del sistema ante perturbaciones, ya que la referencia del control 
es siempre constante, siendo esta el centro de la imagen.

Este proyecto ha sido en gran parte posible por la utilización de la herramienta GIT. Se ha creado un repositorio donde esta alojado toda la información
relevante al proyecto incluyendo los modelos 3D, los códigos y el historial de commits. El proyecto estará liberado bajo la licencia BSD, por tanto esta 
disponible a toda persona interesada en mejorar y contribuir al proyecto o para cualquier uso.


%La hoja de estilo utilizada es una versión de la que el Prof. Payán realizó para un libro que desde hace tiempo viene escribiendo para su asignatura. Con ella se han realizado estas notas, a modo de instrucciones, añadiéndole el diseño de la portada. El diseño de la portada está basado en el que el prof. Fernando García García, de nuestra universidad, hiciera para los libros de la sección de publicación de nuestra Escuela.


\chapter*{Abstract}
\pagestyle{especial}
\chaptermark{Abstract}
\phantomsection
\addcontentsline{toc}{listasf}{Abstract}

\lettrine[lraise=-0.1, lines=2, loversize=0.2]{T}{h}is project presents a 3 degrees of freedom robot made up of a "pan-tilt" mechanism
mounted on top of a rail allowing lateral displacement. This robot is able to visualize
using a 2D camera, any object in its field of view and track it, while centering the object in the center of the image. This is achieved by 
controlling the motors by means of a PID controller, where the signal to control is the position of the object in the image and the error is the 
difference between its position and the center of the image.

To carry out this work, a robot such as the one described above was needed, therefore,
I've take the liberty to design, simulate in CAD, manufacture and assemble the robot to spec for this application.
This document goes into depth in the mechanical design, explaining the functioning of the different joints
and the electronic system that drives the motors and sensors of the system. In addition,
programming of motor control by interrupts and timers has been carried out, allowing a
efficient operation of the embedded system.

As you know, the task of tracking objects using machine vision is expensive in terms of computing power, therefore tracking programs have been 
performed on a PC to obtain acceptable performance. Consequently the need arises to communicate the microcontroller and the PC quickly and reliably. 
This communication was implemented using a state machine that allows managing information packets
that come from the PC, this helps to eliminate false or corrupt information to be executed by the microcontroller.

With all the above complete, only control remains. The control was implemented in the PC in the same loop that takes care of the
tracking, here we have adjusted the gains $Kp$, $Ki$, $Kd$ experimentally to obtain a
good response of the system to disturbances, since the control reference is always constant, this being the center of the image.

This project has been largely possible due to the use of the GIT tool. All the relevant information is stored in a public repositoy, including 3D models,
codes and commit history. The project will be released under the BSD license, therefore it is available
to any person interested in improving and contributing to the project or for any use.



 

% Índice abreviado 
% El índice abreviado se incluye también en algunos libros, con menor detalle que el completo. Descomentar las siguientes líneas.
\cleardoublepage
\phantomsection
\addcontentsline{toc}{listasf}{Índice Abreviado}
\pagestyle{especial}
\shorttoc{Índice Abreviado}{1}

%Índice normal, el completo
\cleardoublepage
\phantomsection
\pagestyle{especial}
\tableofcontents


%:---------------------------Notación 
%Toda esta notación es opcional, pero creemos que puede ser de mucha ayuda.
%Juan José Murillo Fuentes y Javier Payán Somet. Copyright 2011. Todos los derechos reservados.

%:---------------------------------------------------  Referencias
%Puede usar los comandos \label y \ref, pero con lo de abajo se facilita el uso de múltiples etiquetas para ecuaciones, secciones,...
%Etiquetas:
\newcommand{\LABEQ}[1]{\label{eq:#1}}%\mathtt{[eq:#1]}\qquad %Equación
\newcommand{\LABALG}[1]{\label{alg:#1}}%\mathtt{[lab:#1]}\qquad %Algoritmo
\newcommand{\LABTAB}[1]{\label{tab:#1}}%{\tt [tab:$\text{$#1$}$]}} %Tabla
\newcommand{\LABFIG}[1]{\label{fig:#1}}%{\tt [fig:$\text{$#1$}$]}} %Figura
\newcommand{\LABTHM}[1]{\label{thm:#1}}%{\tt [thm:#1]}} % Teorema
\newcommand{\LABPRP}[1]{\label{prp:#1}}%{\tt [prp:#1]}} % Proposición
\newcommand{\LABLEM}[1]{\label{lem:#1}}%{\tt [lem:#1]}} % Lema
\newcommand{\LABCOR}[1]{\label{cor:#1}}%{\tt [cor:#1]}} %Corolario 
\newcommand{\LABDFN}[1]{\label{dfn:#1}}%{\tt [dfn:#1]}} %Definición
%\newcommand{\LABFNT}[1]{\label{fnt:#1}}%{\tt [fnt:#1]}} %
%
%Referencias a las etiquetas anteriores, Incluyen el título. Puede cambiarlos aquí. Por ejemplo, si quiere "Fig." en vez de "Figura"...
\newcommand{\EQ}[1]{\eqref{eq:#1}}%$^{\text{\tt [#1]}}$} %used to be {(\ref{eq:#1})}
\newcommand{\ALG}[1]{~\ref{alg:#1}}
%\newcommand{\TAB}[1]{Tabla ~\ref{tab:#1}}%$^{\text{\tt [#1]}}$}
\newcommand{\TAB}[1]{\autoref{tab:#1}}%$^{\text{\tt [#1]}}$}
%\newcommand{\FIG}[1]{Figura~\ref{fig:#1}} %$^{\text{\tt [#1]}}$} 
\newcommand{\FIG}[1]{\autoref{fig:#1}} %$^{\text{\tt [#1]}}$} 

%\newcommand{\FIG}[1]{Fig. \ref{fig:#1}} %$^{\text{\tt [#1]}}$} 
\newcommand{\THM}[1]{Teorema~\ref{thm:#1}}%$^{\text{\tt [#1]}}$}
\newcommand{\COR}[1]{Corolario~\ref{cor:#1}}%$^{\text{\tt [#1]}}$}
\newcommand{\PRP}[1]{Propiedad~\ref{prp:#1}}%$^{\text{\tt [#1]}}$}
\newcommand{\LEM}[1]{Lema~\ref{lem:#1}}%$^{\text{\tt [#1]}}$}
\newcommand{\DFN}[1]{Definición~\ref{dfn:#1}}%$^{\text{\tt [#1]}}$}
%\newcommand{\FNT}[1]{~\ref{fnt:#1}}%$^{\text{\tt [#1]}}$}

%Etiquetas para títulos tipo capítulo, sección, subsección,...
\newcommand{\LABCHAP}[1]{\label{chap:#1}}%{\tt [chap:#1]}}
\newcommand{\LABAPEN}[1]{\label{apen:#1}}%{\tt [chap:#1]}}
\newcommand{\LABSEC}[1]{\label{sec:#1}}%{\tt [sec:#1]}}
\newcommand{\LABSSEC}[1]{\label{ssec:#1}}%{\tt [ssec:#1]}}
\newcommand{\LABSSSEC}[1]{\label{sssec:#1}}%{\tt [sssec:#1]}}
%
%Referencias para los anteriores títulos
\newcommand{\CHAP}[1]{Capítulo~\ref{chap:#1}}%$^{\text{\tt [c:#1]}}$}
\newcommand{\SEC}[1]{Sección~\ref{sec:#1}}%$^{\text{\tt [s:#1]}}$}
\newcommand{\SSEC}[1]{Subsección~\ref{ssec:#1}}%$^{\text{\tt [ss:#1]}}$}
\newcommand{\SSSEC}[1]{Apartado~\ref{sssec:#1}}%$^{\text{\tt [sss:#1]}}$}
\newcommand{\APEN}[1]{Apéndice~\ref{apen:#1}}%$^{\text{\tt [c:#1]}}$}
%
%%
%\newcommand{\PAGEEQ}[1]{~\pageref{eq:#1}}
%\newcommand{\PAGETAB}[1]{~\pageref{tab:#1}}
%\newcommand{\PAGEFIG}[1]{~\pageref{fig:#1}}


%%%%Definiendo caligrafías especiales 
%\newcommand{\emphb}[1]{\emph{\textbf{#1}}}
%\newcommand{\X}{\calg{X}} %{\ensuremath{\calg{X}}} %\textrm{\ifmmode {1pt} \else {\, } \fi}
%\newcommand{\Y}{\calg{Y}}%{\ensuremath{\calg{Y}}}
\newcommand{\calg}[1]{\ensuremath{\mathcal{#1}}} %JJMF: No entiendo para qué es esto.
%\newcommand{\hb}[1]{\ensuremath{\textrm{\usefont{T1}{phv}{b}{n}#1}}}
%\newcommand{\hn}[1]{\ensuremath{\textrm{\usefont{T1}{phv}{m}{n}#1}}}

%
%:Renombrando overline
%\newcommand{\overl}[1]{\bar{#1}}

\DeclarePairedDelimiter\ceil{\lceil}{\rceil}
\DeclarePairedDelimiter\floor{\lfloor}{\rfloor}

%%%% vectores
%
\newcommand{\vect}[1]{\mathbf{#1}}     %vectors (bold type)
\newcommand{\vc}[1]{\mathbf{#1}}     %vectors (bold type)
\newcommand{\matr}[1]{\mathbf{#1}}     %matrices (bold type)
% ó:
%\newcommand{\vc}[1]{\ensuremath{\mathbf{#1}}}
% ó:
%\newcommand{\vct}[1]{\boldsymbol{#1}}
%\newcommand{\vect}[1]{\boldsymbol{#1}}  %vectors (bold type)
%\newcommand{\matr}[1]{\boldsymbol{#1}}  %matrices (bold type)

\renewcommand*{\j}{\ensuremath{\textrm{j}}}%{\mathop{}\mathrm{j}}

%%%% Complejos y exponenciales
\newcommand{\xp}[1]{\e^{\j{#1}}}         %simple exponential
\newcommand{\xppi}[1]{\e^{\j2\pi{#1}}}         %simple exponential
\newcommand{\nxp}[1]{\e^{-\j{#1}}}       %negative exponential
\newcommand{\nxppi}[1]{\e^{-\j2\pi{#1}}}       %negative exponential
\newcommand{\e}{\mathrm{e}}
%\newcommand{\xp}[1]{\e^{j{#1}}}         %simple exponential
%\newcommand{\nxp}[1]{\e^{-j{#1}}}       %negative exponential
%
%
% Parte real
\newcommand{\re}{\mbox{$\mathrm{I\!Re}$}}       %real part
%\renewcommand{\Re}{\ensuremath{\boldsymbol{\mathcal{R}}}}
%\renewcommand{\Re}{\ensuremath{\textrm{\usefont{T1}{phv}{m}{n}Re}}}
% Parte imaginaria
\newcommand{\im}{\mbox{$\mathrm{I\!Im}$}}       %imaginary part
%\renewcommand{\Im}{\ensuremath{\boldsymbol{\mathcal{I}}}}
%\renewcommand{\Im}{\ensuremath{\textrm{\usefont{T1}{phv}{m}{n}Im}}}
%:Creando la unidad imaginaria
%\renewcommand{\j}{\ensuremath{\textrm{\usefont{T1}{lmr}{m}{n}j}}}


%%%% Maths functions and symbols
%
%:Para definir funciones matemáticas en castellano
\makeatletter
\ifdtsc@English
	\DeclareMathOperator{\sen}{sin}
	\DeclareMathOperator{\tg}{tg}  %tg() function
	\DeclareMathOperator{\arctg}{arctg}     %arctg() function
\fi
\makeatother

\DeclareMathOperator{\sa}{Sa}
%
\DeclareMathOperator{\sgn}{sgn}
%\newcommand{\sgn}{\mathrm{sign}}        %sign() function
%\newcommand{\sign}{\mathrm{sign}}
%
\DeclareMathOperator{\rect}{rect}
\DeclareMathOperator{\sinc}{Sinc}
%\newcommand{\cost}{\psi}                %cost or contrast function
\newcommand{\pder}[2]{\frac{\partial #1}{\partial #2}}  %partial derivative
%
%
%\renewcommand{\mod}{\bmod}      %\:\text{mod}\:}
\newcommand{\RR}{\mathbb{R}}            %real numbers
\newcommand{\CC}{\mathbb{C}}    %complex numbers
%
%\newcommand{\tg}{\mahtrm{tg}}           
%\newcommand{\angl}{\arg}
%
\newcommand{\costo}[2]{\cos^{#1}\!#2}   %cos to power
\newcommand{\sento}[2]{\sin^{#1}\!#2}   %sen to power
%
\newcommand{\gra}{\ensuremath{^\circ}}  %Grados. Ejemplo: $5\gra$ K serían 5º K


%%%% Matrices, traspuesta, hermítica, ...
%
\newcommand{\inv}{^{-1}}                %inverse operator
\newcommand{\trs}{^\top}                %transposition operator
%\newcommand{\trs}{^{\textrm{\usefont{T1}{phv}{b}{n}{T}}}}          %transponer una matriz
\newcommand{\psd}{^\dagger}             %pseudoinverse operator
\newcommand{\cnj}{^*}                   %complex conjugate
\newcommand{\pcnj}{^{\phantom{*}}}      %phantom complex conjugate (for alignment)
\newcommand{\her}{^\mathrm{H}}          %complex conjugate transpose
%\newcommand{\her}{^{\textrm{\usefont{T1}{phv}{b}{n}{H}}}}          %Hermítica
\newcommand{\id}[1]{\vect{I}_{#1}}       %identity matrix
%\newcommand{\id}{\matr{I}}       %identity matrix
\newcommand{\diag}[1]{\mathrm{diag}\left(#1\right)}     %diagonal
%\DeclareMathOperator{\diag}{diag}



%%%% indices de prestaciones %%%%%%%%%
%
%\newcommand{\isr}{\mathrm{ISR}}           %interference-to-signal ratio
\newcommand{\snr}{\mathrm{SNR}}           %signal-to-noise ratio
\newcommand{\mse}{\mathrm{MSE}}           %minimum mean square error
%
%ó se pueden escribir como
%\newcommand{\SNR}{\ensuremath{\textrm{SNR}}}
%...


%%%%Miscellaneos
%
%Redefiniendo epsilon
%\renewcommand{\epsilon}{\ensuremath{\textrm{\usefont{OML}{cmr}{m}{n}\symbol{15}}}}
%
%:Creando ``tal que'' de las expresiones matemáticas
\newcommand{\talq}{\colon}
%
%%Creando ``igual por definición'' de las expresiones matemáticas
\newcommand{\eqdef}{\ensuremath{\mathrel{\stackrel{\mathrm{def}}{=}}}}
%%ó
%\newcommand{\eqdef}{\triangleq}         %equal by definition
%
%
%:Creando la igualdad basada en una ecuación
%\newcommand{\igualref}[1]{\ensuremath{\mathrel{\stackrel{\mathrm{\eqref {#1}}}{=}}}}
%
%:Definiendo cardinal y norma
\newcommand{\norm}[1]{\ensuremath{\left\lVert #1 \right\rVert }}
\newcommand{\card}[1]{\ensuremath{\left| #1\right|}}
%\newcommand{\card}[1]{\ensuremath{\text{card}~#1}
%
%:Renombrando \boldsymbol
%\newcommand{\bm}[1]{\boldsymbol{#1}}
%
%:Facilitando la escritura de X_{i}
\newcommand{\xyz}[3]{\ensuremath{#1_{#2},#2=1,2,\ldots,#3}}
%
%:Creando el diferencial. Por defecto, dx
%\newcommand*{\df}[1][x]{\mathop{}\!\mathrm{d}{#1}}
\newcommand{\df}[1]{\mathrm{d}{#1}}

%
%%Modificando el menor igual y el mayor igual
\renewcommand{\le}{\leqslant}           %fancy \le
\renewcommand{\ge}{\geqslant}           %fancy \ge
%
%:Creando BL=backslash de las expresiones matemáticas
\newcommand{\BL}{\ensuremath{\backslash}}
%
%Redefiniendo iff
\renewcommand{\iff}{\Leftrightarrow}
%
%\newcommand{\what}{\widehat}
%\newcommand{\supp}[1]{^{(#1)}}          %superindex with parentheses
\newcommand{\eqexpl}[1]{\underset{#1}{\underset{\uparrow}{=}}}  %equal with explanation
%\newcommand{\proof}{\noindent {\bf Proof.} }
%\newcommand{\skproof}{\noindent {\bf Sketch of the proof.} }
\newcommand{\sfrac}[2]{\tfrac{#1}{#2}}  %small frac
\newcommand{\inc}{\Delta}
\newcommand{\ten}[1]{\cdot 10^{#1}}     %scientific notation
%\newcommand{\arrow}{$\rightarrow$ }
%\newcommand{\darrow}{$\Rightarrow$ }
%\newcommand{\tends}{\rightarrow}                         %'tends to'
\newcommand{\tendsub}[1]{\xrightarrow[#1]{}}             %{\underset{#1}{\longrightarrow}} %'tends to' with subscript
%\newcommand{\tendsubsup}[2]{\xrightarrow[#1]{#2}}        %{\overset{#2}{\tendsub{#1}}}   %'tends to' with sub and superscript
\newcommand{\ord}{\mathrm{O}}         %order of magnitude
\newcommand{\tm}{^{\text{\tiny{TM}}}} %trademark
%
%
%:Creando ``L'' de las expresiones matemáticas
%\renewcommand{\L}[1][L\!]{\ensuremath{\boldsymbol{\mathcal{#1}}}}
%\renewcommand{\L}[1][L\!]{\ensuremath{\boldsymbol{\mathscr{#1}}}}
%
%:Creando la F de transformada de Fourier
%\newcommand{\Fo}{\ensuremath{\boldsymbol{\mathscr{F}}}}
%
%:Creando la H de transformada de Hilbert
%\newcommand{\Hi}{\ensuremath{\boldsymbol{\mathscr{H}}}}
%
%
%%%% Indention
%
%\newcommand{\ind}{$\phantom{\indent}$}
%


%%%% Basic statistics
%
%%Creando el operador "valor esperado" 
%\newcommand{\E}{\ensuremath{\mathbb{E}}}
%\newcommand{\E}{\ensuremath{\textrm{\usefont{OML}{phv}{b}{n}E\hspace{1pt}}}}
%\newcommand{\E}{\ensuremath{\textrm{\usefont{T1}{phv}{m}{n}E}}}
\newcommand{\E}{\mathbb{E}}             %expected value
%
%%Definiendo el espacio de probabilidad
%\newcommand{\ep}{\ensuremath{\left( {\Omega, \calg{B}, \Pr} \right)}}
%\newcommand{\EP}{\ensuremath{\left( {\Omega, \calg{B}, \Pr} \right)}}
%
%\newcommand{\var}{\mathrm{Var}}         %variance
%\newcommand{\cov}{\mathrm{Cov}}         %covariance
\newcommand{\covm}[1]{\vc{C}_{#1}}           %covariance matrix
\newcommand{\corrm}[1]{\vc{R}_{#1}}           %correlation matrix
%\newcommand{\pdf}{p}                    %probability density function
%
%%Creando el operador "Probabilidad"
%\renewcommand{\Pr}{\ensuremath{\mathbb{P}}}
%\renewcommand{\Pr}{\ensuremath{\textrm{\usefont{T1}{phv}{m}{n}P}}}
%
%%Momentos
%\newcommand{\m}[1]{\mu_{#1}}            %moments
%\newcommand{\K}[1]{\kappa_{#1}}         %cumulants (symbol)
%\newcommand{\eK}[1]{\hat{\kappa}_{#1}}  %estimated cumulant
%\newcommand{\cum}{\mathrm{Cum}}         %cumulant
%\newcommand{\M}{\mathrm{M}}             %moment
%\newcommand{\C}{\mathrm{C}}             %cumulant (short)
%
%Creando la varianza
\newcommand{\si}[1]{\ensuremath{\sigma_{#1}^{2}}}
%
%Creando la expresión para indicar una gaussiana
\newcommand{\gauss}[2]{\ensuremath{\calg{N}\left( {#1, #2} \right)}}
%


%%Creando el espectro del ruido blanco
%\newcommand{\Sw}[1][W]{\ensuremath{S_{#1}\left( {\omega} \right)  = \frac{N_{0}}{2}}}
%\newcommand{\Sf}[1][W]{\ensuremath{S_{#1}\left( {\omega} \right)  = \frac{N_{0}}{2}}}
%\newcommand{\Swu}[1][W]{\ensuremath{S_{#1}\left( {\omega} \right)  = N_{0}/2} \textrm{w/(rad/s)}\finjps}
%\newcommand{\Sfu}[1][W]{\ensuremath{S_{#1}\left( {\omega} \right)  = N_{0}/2} \textrm{w/(Hz)}\finjps}
%
%%Creando el límite en el sentido de error cuadrático medio. Está copiado de amsopn.sty
%\def\lms{\qopname\relax m{l.i.m.}}
%
%%Creando el conjunto típico
%\newcommand{\ct}[1][T]{\ensuremath{#1_{\epsilon}^{n}}\ifmmode \else \ \fi}%[A]{\ensuremath{#1_{\epsilon}^{\left( {n} \right)}}}
%
%%Creando la función de distribución. Por defecto, F_{X}\left( {x} \right)
%\newcommand{\FD}[1][x]{\ensuremath{F_{\MakeUppercase{#1}}\left( \MakeLowercase{#1} \right)\ }}
%\newcommand{\FDP}{función de distribución\ }
%
%%Creando la función de densidad de probabilidad. Por defecto, f_{X}\left( {x} \right)
%\newcommand{\fd}[1][x]{\ensuremath{f_{\MakeUppercase{#1}}\left( \MakeLowercase{#1} \right)\ }}
%\newcommand{\fdp}{función densidad de probabilidad\ }



%%%% Revisiones
%
%\newcommand{\comentario}[1]{{\bf Comentario:} {\tt #1}?} 

%\newcommand{\cambiopor}[3]{\marginpar{\hfill{$\tendsubsup{#2}{\text{\tt #1}}$}} {\tt >>> #3}}
%
%\newcommand{\cambio}[2]{{{\tt #1}} {\tt >>> #2??}}
%
%\newcommand{\incluir}[1]{{\bf Incluir:} {\tt #1}?}
%\newcommand{\incluido}[1]{{\bf Incluido:} {\tt #1}}
%
%\newcommand{\notaFul}[1]{{\color{blue} {Fulano: \bf #1}}}



%%%%Ejemplo de notación típica
%
%\def\w{{\mathbf w}}        %GP Vector
%\def\r{{\mathbf r}}        %
%\newcommand{\PHI}{\boldsymbol{\Phi}}
%\def\b{{b}} %elemento de b
%\def\bve{{\mathbf \bve}}
%\def\d{{d}}
%\def\k{{ k}}
%\def\kk{{\mathbf \k}}
%\def\K{{\mathbf K}}
%\def\x{{\vect{x}}}
%\def\y{\vect{\b}}
%\def\newp{_{*}}        %GP Vector
%\def\newout{\b_{*}}
%\newcommand{\p}{\boldsymbol{\phi}}
%\def\X{{\mathbf X}}
%\def\teta{{\mathbf \theta}}
%\def\muw{{\mathbf \mu_\w}}
%\def\newin{{\vect{x}\subind{*}}}
%\def\fv{f}
%\def\fp{\vect{f}}
%
%\def\tset{\mathcal{D}}
%\def\mgp{\boldsymbol{\mu}}
%\def\Nor{\mathcal{N}}                %Gaussian distribution
%\def\treg{y} %Target value of a regression problem
%\def\tregnew{y_{*}}
%
%\ifluatex
%\makeatletter
%\DeclareRobustCommand{\LaTeX}{L\kern-.36em%
%        {\sbox\z@ T%
%         \vbox to\ht\z@{\hbox{\check@mathfonts
%                              \fontsize\sf@size\z@
%                              \math@fontsfalse\selectfont
%                              A}%
%                        \vss}%
%        }%
%        \kern-.15em%
%        \TeX \xspace}
%\makeatother
%\else
%%	\renewcommand{\LaTeX}{LaTeX\xspace}
%%	\renewcommand{\TeX}{TeX\xspace}
%\fi
%

%%:Una posible propuesta de logos
\usepackage{hologo}[2016/05/16]
\protected\def\latex{%
  L%
  \mbox{\kern-.35em\raisebox{0.348ex}{\scalebox{0.75}{A}}\kern-.15em \TeX}\hspace{-0.15em}}
  
\protected\def\lualatex{%
  L%
  \mbox{\kern-.38em\raisebox{0.348ex}{\scalebox{0.75}{U\kern-.16em A}}\kern-.09em \latex}}

\protected\def\pdflatex{%
  P%
  \mbox{\kern-.15em\raisebox{0ex}{\scalebox{0.75}{d\kern-.1em f}}\kern-.05em \latex}}

\protected\def\LuaLaTeX{%
  L%
  \mbox{\kern-.38em\raisebox{0.348ex}{\scalebox{0.75}{U\kern-.16em A}}\kern-.09em \latex}}

\protected\def\PdfLaTeX{%
  P%
  \mbox{\kern-.15em\raisebox{0ex}{\scalebox{0.75}{d\kern-.1em f}}\kern-.05em \latex}}

\protect\def\XeLaTeX{%
\hologo{Xe}%
   L%
  \mbox{\kern-.35em\raisebox{0.348ex}{\scalebox{0.75}{A}}\kern-.15em \TeX}\hspace{-0.15em}}

\protect\def\xelatex{%
\hologo{Xe}%
   L%
  \mbox{\kern-.35em\raisebox{0.348ex}{\scalebox{0.75}{A}}\kern-.15em \TeX}\hspace{-0.15em}}




 %No incluir si no se quiere, comentándolo

%:Empieza el contenido del libro
\mainmatter

%:Página por defecto
\pagestyle{esitscCD}

%:Los diferentes capítulos, en carpetas separadas
% % !TEX root =../LibroTipoETSI.tex
%El anterior comando permite compilar este documento llamando al documento raíz
\chapter{Instrucciones para la Cubierta y la Portada}\label{chp-01}
\epigraph{Aunque aquí se incluye la descripción de la cubierta y portada aprobadas por la ETSI, el presente formato está preparado para que introduzca los datos necesarios en el fichero principal, \ttcolor{pfcTipoETSI.tex}, y el compilador genere automáticamente la cubierta y portada siguiendo las directrices aprobadas.}%{Claude Shannon, 1948}

%\lettrine[lraise=0.7, lines=1, loversize=-0.25]{E}{n}
\lettrine[lraise=-0.1, lines=2, loversize=0.2]{L}{a} cubierta es la tapa del proyecto, mientras que la portada es la primera hoja que aparece al abrirlo. En Junta de Escuela de 25 de abril de 2014 se aprobó la obligatoriedad de utilizar la cubierta y portada que se incluyen en este ejemplo de formato y siguiendo las siguientes instrucciones. Debe modificar, en su caso y para la cubierta,
\begin{itemize}\itemsep1pt \parskip0pt \parsep0pt
\item la titulación, 
\item	el tipo de proyecto, atendiendo a si es fin de carrera, grado o máster.
\item el título del proyecto, 
\item el autor,
\item	el tutor o tutores,
\item el departamento,
\item y la fecha (año).
\end{itemize}

Para la portada, además de los anteriores, deberá cambiar el cargo del tutor. Por otro lado si el tutor no es docente de la ETSI entonces tendrá que añadir la figura de tutor ponente, que es un profesor de la ETSI encargado de realizar la gestión de la defensa.
El proyecto se escribirá en A4. En la cubierta, los diferentes campos se localizarán siguiendo el ejemplo de la cubierta en este documento. Tendrán los tamaños de letras y la posición orientativa siguientes, esta última dada en coordenadas en cm tomando como referencia la esquina inferior izquierda,
\begin{itemize}\itemsep1pt \parskip0pt \parsep0pt
\item la titulación 21 pt, (4.2,27)
el tipo de proyecto 21 pt, (4.2,25.9)
\item el título del proyecto 21 pt, (4.2,16.6), podrá subirse si el título excede dos líneas
\item el autor y tutor/es 15 pt, (4.2,13)
 \item el departamento, nombre de la ETSI y de la US, 14 pt y negrita, (centrado, 7.8). Si el nombre del departamento no cupiese en una línea, se utilizaría la siguiente, desplazando el texto inferior convenientemente.
\item para el texto “Sevilla, año” 13 pt, (centrado, 5.5)
\end{itemize}

Para la cubierta 
\begin{itemize}\itemsep1pt \parskip0pt \parsep0pt
\item la titulación 14 pt, (centrado,27.5)
\item el tipo de proyecto 14 pt, (centrado,26.9)
\item el título del proyecto 21 pt y negrita, (centrado,23)
\item el autor y tutor/es 11 pt, (centrado,19.5)
\item el departamento, nombre de la ETSI y de la US, 14 pt, (centrado 13)
\item para el texto “Sevilla, año” 11 pt, (centrado, 10.4)
\end{itemize}

La cubierta deberá incluir la imagen de fondo que se incluye en la cubierta de este texto, con las dos bandas vertical y horizontal en el color de la fachada del edificio Plaza América y la pequeña imagen de un detalle del edificio en la zona de cruce de las bandas. Incluirá el logotipo de la ETSI a la derecha del nombre de departamento. A pie de cubierta aparecerá el logo de la Universidad de Sevilla. El logo del departamento es opcional. Si no se incluyese, el de la Universidad de Sevilla se centraría en la hoja.

En la Sección \ref{sec:guía.cubierta}, encontrará alguna indicación de cómo puede modificar el aspecto de la portada. \textbf{En cualquier caso, el formato está preparado para que introduzca los datos necesarios en el fichero principal, \ttcolor{pfcTipoETSI.tex}, y el compilador genere automáticamente la cubierta y portada}.


\endinput

%
\include{introduccion/introduccion}%Guia uso
%
\include{Capitulo1/Capitulo1}
\include{Capitulo2/Capitulo2}
\include{Capitulo3/Capitulo3}
\include{Capitulo4/Capitulo4}
\include{Capitulo5/Capitulo5}
\include{Capitulo6/Capitulo6}
\include{Capitulo7/Capitulo7}
%
%\include{CapituloLibroETSI/CapituloLibroETSI} 
%
%\include{CapituloProblemasLibroETSI/CapituloProblemasETSI} 

%\include{CapituloEstilo/estilolibroetsi}
% 
%%:Empezamos con los apéndices, que irían en uno o más ficheros. Es necesario incluir estos ficheros entre el entorno \begin{appendices}....\end{appendices} debido a que se ha deseado utilizar un formato diferente para el título de los apéndices, incluyendo la palabra apéndice, para la numeración de los apéndices, alfabético, y para las cabeceras de las páginas.
%
\begin{appendices}
%
%% Fichero en el que se incluyen los apéndices
% !TEX root =../LibroTipoETSI.tex



%APENDICE A
\chapter{Sobre  \LaTeX}\LABAPEN{ApA}
{Este es un ejemplo de apéndices, el texto es únicamente relleno, para que el lector pueda observar cómo se utiliza}
%%%%%%%%%%%%%%%%%
\section{Ventajas de \LaTeX}

El gusto por el \LaTeX\ depende de la forma de trabajar de cada uno. La principal virtud es la facilidad de formatear cualquier texto y la robustez. Incluir títulos, referencias es inmediato.
%\Blindtext
%\lipsum
Las ecuaciones quedan estupendamente, como puede verse en \EQ{Ap1}
\begin{equation}\LABEQ{Ap1}
x_{1}=x_{2}.
\end{equation}


\section{Inconvenientes}
%\Blindtext
El principal inconveniente de \LaTeX\ radica en la necesidad de aprender un conjunto de comandos para generar los elementos que queremos. Cuando se está acostumbrado a un entorno ``como lo escribo se obtiene'', a veces resulta difícil dar el salto a ``ver'' que es lo que se va a obtener con un determinado comando. 

Por otro lado, en general será muy complicado cambiar el formato para desviarnos de la idea original de sus creadores. No es imposible, pero sí muy difícil. Por ejemplo, con la sentencia siguiente:
 
\begin{lstlisting}[language=,caption={Escritura de una ecuación}, breaklines=true, label=prgA1-01]
\begin{equation}\LABEQ{Ap2}
x_{1}=x_{2}
\end{equation}
\end{lstlisting}
obtenemos:
\begin{equation}\LABEQ{Ap2}
x_{1}=x_{2}
\end{equation}
Esto será siempre así. Aunque, tal vez, esto podría ser una ventaja y no un incoonveniente.

Para una discusión similar sobre el Word\tsp{\textregistered}, ver \APEN{ApB}.
%\Blindtext


%%%%%%%%%%%%%%%%%%%%%%%%%%%%%%%%%%%%%%%
%APENDICE B
\chapter{Sobre Microsoft Word\tsp{\textregistered}}\LABAPEN{ApB}

\section{Ventajas del Word\tsp{\textregistered}}
La ventaja mayor del Word\tsp{\textregistered} es que permite configurar el formato muy fácilmente. Para las ecuaciones,
\begin{equation}
x_{1}=x_{2},
\end{equation}
tradicionalmente ha proporcionado pésima presentación. Sin embargo, el software adicional Mathtype\tsp{\textregistered} solventó este problema, incluyendo una apariencia muy profesional y cuidada. Incluso permitía utilizar un estilo similar al \LaTeX\xspace. Además, aunque el Word\tsp{\textregistered} incluye sus propios atajos para escribir ecuaciones,  Mathtype\tsp{\textregistered} admite también escritura \LaTeX\xspace. En las últimas versiones de Word\tsp{\textregistered}, sin embargo, el formato de ecuaciones está muy cuidado, con un aspecto similar al de \LaTeX.


\section{Inconvenientes de Word\tsp{\textregistered}}
Trabajar con títulos, referencias cruzadas e índices es un engorro, por no decir nada sobre la creación de una tabla de contenidos. Resulta muy frecuente que alguna referencia quede pérdida o huérfana y aparezca un mensaje en negrita indicando que  no se encuentra. 

Los estilos permiten trabajar bien definiendo la apariencia, pero también puede desembocar en un descontrolado incremento de los mismos. Además, es muy probable que Word\tsp{\textregistered} se quede colgado, sobre todo al trabajar con copiar y pegar de otros textos y cuando se utilizan ficheros de gran extensión, como es el caso de un libro.

%\end{equation}
 %Ver este fichero para incluir ahí los apéndices.
%
\end{appendices}
%:Fin de la inclusión de apéndices

%:Empieza todo lo que no constituye el cuerpo en si del libro. Todo lo que va detrás
\backmatter

%:Indice de figuras, coméntese las siguientes líneas si no se desea
\cleardoublepage
\phantomsection

%:Para añadir una línea en blanco en el TOC y separar esta lista
\addtocontents{toc}{\protect\mbox{}\protect\hspace*{0pt}\par}
\addcontentsline{toc}{listasb}{\listfigurename}
\pagestyle{especial}
\listoffigures

%:Indice de tablas, coméntese las siguientes líneas si no se desea
\cleardoublepage
\phantomsection
\addcontentsline{toc}{listasb}{\listtablename}
\pagestyle{especial}
\listoftables

%:Indice de Programas
\cleardoublepage
\phantomsection
\addcontentsline{toc}{listasb}{\lstlistlistingname}
\pagestyle{especial}
\lstlistoflistings

%:Bibliografía con biblatex y biber
\cleardoublepage
\phantomsection
\addcontentsline{toc}{listasb}{\bibname}
\pagestyle{especial}
%BIBER
%\printbibliography[heading=etsi]
%BIBTEX
%\bibliographystyle{IEEEtran}
\bibliographystyle{amsplain} %flexbib amsplain alpha
%:Fichero con la bibliografía, BIBTEX
\bibliography{bibliografiaLibroETSI}

%:Índice alfabético de palabras
\cleardoublepage
\phantomsection
\addcontentsline{toc}{listasb}{\indexname}
\chaptermark{\indexname}
\printindex


%:Acrónimos
\cleardoublepage
\phantomsection
\addcontentsline{toc}{listasb}{\glossaryname}
\chaptermark{\glossaryname}
\printglossaries

\end{document}