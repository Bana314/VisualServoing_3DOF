% !TEX root =../LibroTipoETSI.tex



%APENDICE A
\chapter{ }\LABAPEN{ApA}
{}
%%%%%%%%%%%%%%%%%

\subsection{Encoders magneticos}

Para los ejes del mecanismo \indexit{"pan-tilt"} se optó por incorporar un tipo de sensores usados ampliamente en la róbotica, 
los sensores magnéticos de effecto hall. En la figura \FIG{as5047} se muestra un diagrama del sensor elegido con una 
ilustración del su funcionamiento. Sin embargo no se llego a usar dado que podemos contar en bucle abierto la posición exacta de 
los motores. Aún asi se podrían incorporar más adelante para detectar si ha habido alguna colisión o para mover el mecanismo
con los motores deshabilitados y no perder su posición. Esto es útil si queremos entenar al robot varios puntos moviedo el 
mecanismo a mano y que repita ese moviemiento automáticamente.

\begin{figure}[htbp]
    \includegraphics[width=10 cm]{Capitulo3/figuras/as5047.pdf}
    \caption{Sensor montado en eje AS504X}
    \LABFIG{as5047} %Esto es una forma propia de los autores de gestionar las etiquetas y referencias
    \end{figure}

El sensor magnético usado es uno que se monta en el propio eje del motor. Un imán que está magnetizado axialemtne se coloca en 
el rotor del motor o eje móvil cuyo ángulo es el que se quiere medir y se coloca el sensor concentricamente al eje debajo del imán. Este sensor 
esta dispuesto de 4 sensores hall combinadas con un puente de wheatstone, posee circuitería interna que se encarga del acondicionamiento
de las señales del puente, además nos proporciona una interfaz digital por el cual podemos hacer la lectura del ángulo. 

Estos tipos de sensores suelen ser mas caros, por ello y con motivo de aprender la diciplina de diseño de PCBs, se
diseño una placa de adaptación para el sensor. El diseño se ha realizado mediante EasyEDA. El resultado del diseño y su funcionamiento
esta reflejado en la figura. 

\begin{figure}[htbp]
    \includegraphics[width=10 cm]{Capitulo3/figuras/pcb.pdf}
    \caption{PCB para el sensor AS504X}
    \LABFIG{pcb_board} %Esto es una forma propia de los autores de gestionar las etiquetas y referencias
    \end{figure}

En la figura \FIG{pcb_board} vemos una placa de adaptación para el sensor magnético integrado en configuración absoluta con interfáz i2C.
Por tanto podemos leer mediante el bus i2c el valor absoluto de posición de la articulación deseada y como se conecta al bus i2C podemos
conectar varios de estos sensores en cadena ya que esta dispuesto en la pcb unos jumpers para preselecionar la dirección por defecto de dispositvo.


% https://www.roboticsbible.com/robot-mechanical-transmission-systems.html

% https://www.elprocus.com/stepper-motor-types-advantages-applications/

% https://www.nema.org/Standards/SecureDocuments/ICS16.pdf

% Instrument Engineers' Handbook, Vol. 2: Process Control and Optimization, 4th Edition. Ed. Liptak, Bela G, N.p.: CRC Press. Print

% https://www.arduino.cc/

% https://www.pololu.com/product/2980

% https://www.electronicshub.org/basics-uart-communication/

% https://copperhilltech.com/content/Application%20Note%20-%20Arduino%20Due%20Timer%20Control.pdf

% https://asf.microchip.com/docs/latest/sam3x/html/asfdoc_sam_drivers_tc_qsg.html

% http://ww1.microchip.com/downloads/en/AppNotes/doc8017.pdf

% https://www.youtube.com/watch?v=fHAO7SW-SZI&t=3s

% http://ww1.microchip.com/downloads/en/DeviceDoc/Atmel-11057-32-bit-Cortex-M3-Microcontroller-SAM3X-SAM3A_Datasheet.pdf

% https://asf.microchip.com/docs/latest/sam3x/html/group__asfdoc__sam__drivers__tc__group.html

% http://ww1.microchip.com/downloads/en/AppNotes/Atmel-42301-SAM3-4S-4L-4E-4N-4CM-4C-G-Timer-Counter-TC-Driver_ApplicationNote_AT07898.pdf

% P. Corke. Dynamic issues in robot visual-servo systems. In International Symposium on Robotics Research, pages 488–498, 1995.   

% F. Chaumette. Potential problems of stability and convergence on image-based and position-based visual servoing. In D. Kriegman, G. Hagar, and S. Morse, editors, Con- fluence of Vision and Control, Lecture Notes in Control and Information Systems, volume 237, pages 66–78. Springer-Verlag, 1998

% J. T. Feddema and O. R. Mitchell. Vision-guided servoing with feature-based tra- jectory generation. IEEE Transactions on Robotics and Automation, 5(5):691–700, October 1989

% Emilio Maggio; Andrea Cavallaro (2010). Video Tracking: Theory and Practice. 1. ISBN 9780132702348. "Video Tracking provides a comprehensive treatment of the fundamental aspects of algorithm and application development for the task of estimating, over time."

% J.F. Henriques, et al. “High-Speed Tracking with Kernelized Correlation Filters.” IEEE Transactions on Pattern Analysis and Machine Intelligence, 37 (3) (2015), pp. 583-596
